% Created 2017-02-11 Sat 00:49
% Intended LaTeX compiler: pdflatex
\documentclass[11pt]{article}
\usepackage[utf8]{inputenc}
\usepackage[T1]{fontenc}
\usepackage{graphicx}
\usepackage{grffile}
\usepackage{longtable}
\usepackage{wrapfig}
\usepackage{rotating}
\usepackage[normalem]{ulem}
\usepackage{amsmath}
\usepackage{textcomp}
\usepackage{amssymb}
\usepackage{capt-of}
\usepackage{hyperref}
\author{W-M}
\date{\today}
\title{}
\hypersetup{
 pdfauthor={W-M},
 pdftitle={},
 pdfkeywords={},
 pdfsubject={},
 pdfcreator={Emacs 24.5.1 (Org mode 9.0.2)}, 
 pdflang={English}}
\begin{document}

\tableofcontents

The EpicAuth protocol has two separate parts:

\begin{enumerate}
\item Authentication
\item Verification.
\end{enumerate}

As part of Authentication Sequences, a Service might ask for zero or more pieces of Verified information.


\section{Authentication}
\label{sec:org9f71c4c}
\subsection{Players:}
\label{sec:org28718c6}
\subsubsection{User: A single identity of a user. User wants to authenticate with this identity at Service.}
\label{sec:org69064d1}
\begin{enumerate}
\item Has a public/private key pair.
\label{sec:org62c8df5}
\end{enumerate}
\subsubsection{Service: A service that has many users. Users can register/login iff they are able to authenticate here.}
\label{sec:org230d7a7}
\begin{enumerate}
\item Has a public/private key pair.
\label{sec:org4195690}
\end{enumerate}
\subsection{Procedure:}
\label{sec:org835f90b}
\begin{enumerate}
\item User is on webpage of Service, and can choose to authenticate using EpicAuth.
\item -> User clicks button.
\item <- Service sends list of To Be Verefied Datasnippets(TBVDs) to be provided by user.
\item Client-side software shows User what TBVDs are requested. User can decide between:
a) Cancel. Stops process here.
b) User can pick which identity they want to use. (Only ones that contain all required TBVDs are selectable)
  b') After selecting Identity, for each datasnippet, which alternative could be used.
\item -> User sends list of selected VDs to Service.
\item Service checks if VDs are Valid.
\end{enumerate}
a) Invalid. Returns failure response to User.
b) Valid. Continue
\begin{enumerate}
\item Service creates Authentication Token.
\item <- Service returns Authentication Token to Client.
\item Client can decide where to store Authentication Token for future use.
\begin{itemize}
\item Locally
\item In the Blockchain (so other devices that share same Identity(privkey) can now also see/use this authentication token.
\end{itemize}
\item Client has been Authenticated Successfully.
\end{enumerate}

\section{Recurring ('Automatic') Authentication}
\label{sec:orgc966fea}
\subsection{Players:}
\label{sec:orge285d4e}
\subsubsection{User}
\label{sec:orgf90c0e2}
\subsubsection{Service}
\label{sec:org087ce32}
\subsection{Procedure:}
\label{sec:orge2a6785}
\begin{enumerate}
\item User visits webpage of Service.
\item Client-side software sees that he has an Authentication Token for this Service (by checking local storage, by checking Blockchain Identity Contract for references to Authentication Tokens)
\item -> Client sends Authentication Token to Service.
\item Service decrypts Authentication Token and checks if not yet expired.
a) Expired. Procedure stops here. Return failure. (Redirect to normal Authentication Flow and return TBVDs?)
b) Continue.
\item <- Service logs in Client, and returns success.
\end{enumerate}

\section{Verification}
\label{sec:orgd03a8eb}
\subsection{Players:}
\label{sec:orgd801fac}
\subsubsection{User}
\label{sec:orga609c0b}
\subsubsection{Verifier (Service)}
\label{sec:org9ae8d88}
\subsection{Procedure:}
\label{sec:org54e5d3a}
\begin{enumerate}
\item Through some external, verification-type-based means, some information might be verified for a User's Identity by the Verifier.
\item At this time, the Verifier creates a Verification Datasnippet.
\begin{itemize}
\item Optionally, the Service might create a Revocation Token (that is referenced from the Verification Datasnippet). This Revocation Token is put on the Blockchain.
\end{itemize}
\item <- Service sends Verification Token to User.
\item User decides whether to store Verification Token (encrypted!) only locally, or also as part of their Blockchain Identity Contract.
\end{enumerate}

\section{Data Types:}
\label{sec:org138c4d5}
note: Not all of these are stored on the blockchain!
\subsection{List of 'To Be Verified Datasnippets'(TBVDs):}
\label{sec:org392820c}
\subsubsection{Each element of list is (unordered) set of alternatives.}
\label{sec:orgcada34c}
\subsubsection{Each alternative is a \{key, verified\(_{\text{id}}\)\}, where `key` is some (conventionalized) string value, and `verifier\(_{\text{id}}\)` is a reference to the Service that should have verified it.}
\label{sec:orga2ed2f8}
\subsubsection{A special alternative is called 'none', which is used in case of optional datasnippets.}
\label{sec:orgca0485f}
\subsection{Verified Datasnippets (VDs):}
\label{sec:org332eb13}
\subsubsection{Each Datasnippet is: \{key, verifier\(_{\text{id}}\), data, revocation\(_{\text{ref}}\), verifier\(_{\text{signature}}\)\}}
\label{sec:org0d49361}
\begin{enumerate}
\item key: Key that matches TBVD key.
\label{sec:org987a297}
\item data: string-data.
\label{sec:org7dfddd7}
\item verifier\(_{\text{id}}\): ID of verifier (matches with TBVD verifier\(_{\text{id}}\))
\label{sec:org9d5c2fb}
\item revocation\(_{\text{ref}}\): Reference (blockchain address) to location of Revocation Contract. Might be 'null' in case that `verifier` decided that was okay.
\label{sec:org4f95324}
\item verifier\(_{\text{signature}}\): Cryptographic Signature of delimited concatenation of all of above fields.
\label{sec:org7e216fb}
\end{enumerate}
\subsubsection{Validity can be verified by Service by checking if the status of the Revocation Contract is modified to be 'revoked'.}
\label{sec:org16b3576}
\subsection{Revocation Token}
\label{sec:orgc39308b}
\subsubsection{Stored in Blockchain by Service (at the same time that a Verified Datasnippet is provided to a User).}
\label{sec:org5fdba12}
\subsubsection{Contains a list of addresses of Service Contracts or Identity Contracts that are allowed to revoke it at a later time (which might only be Service, but possibly also others).}
\label{sec:org2333475}
\subsubsection{Contains a field that can be changed which is 'valid' at first, and might be changed (once, irreversibly) to 'revoked'.}
\label{sec:org4c5f9d5}
\subsubsection{Is referenced from a Verified Datasnippet.}
\label{sec:org77964cf}
\subsection{Authentication Token}
\label{sec:org57f7cdd}
\subsubsection{Encrypted by Service with public key of Service}
\label{sec:org99fdbdd}
\begin{enumerate}
\item So later, only Service can read it again.
\label{sec:org33876c6}
\end{enumerate}
\subsubsection{Contains timestamp of Authentication, and possibly some (small amount of) other metadata.}
\label{sec:orgd2c2b3e}
\subsubsection{User can decide whether to store this Authentication Token as part of his Identity Contract on the Blockchain, or only store it locally.}
\label{sec:orgdc4ae92}
\end{document}
